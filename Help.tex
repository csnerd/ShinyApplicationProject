% Options for packages loaded elsewhere
\PassOptionsToPackage{unicode}{hyperref}
\PassOptionsToPackage{hyphens}{url}
%
\documentclass[
]{article}
\usepackage{lmodern}
\usepackage{amssymb,amsmath}
\usepackage{ifxetex,ifluatex}
\ifnum 0\ifxetex 1\fi\ifluatex 1\fi=0 % if pdftex
  \usepackage[T1]{fontenc}
  \usepackage[utf8]{inputenc}
  \usepackage{textcomp} % provide euro and other symbols
\else % if luatex or xetex
  \usepackage{unicode-math}
  \defaultfontfeatures{Scale=MatchLowercase}
  \defaultfontfeatures[\rmfamily]{Ligatures=TeX,Scale=1}
\fi
% Use upquote if available, for straight quotes in verbatim environments
\IfFileExists{upquote.sty}{\usepackage{upquote}}{}
\IfFileExists{microtype.sty}{% use microtype if available
  \usepackage[]{microtype}
  \UseMicrotypeSet[protrusion]{basicmath} % disable protrusion for tt fonts
}{}
\makeatletter
\@ifundefined{KOMAClassName}{% if non-KOMA class
  \IfFileExists{parskip.sty}{%
    \usepackage{parskip}
  }{% else
    \setlength{\parindent}{0pt}
    \setlength{\parskip}{6pt plus 2pt minus 1pt}}
}{% if KOMA class
  \KOMAoptions{parskip=half}}
\makeatother
\usepackage{xcolor}
\IfFileExists{xurl.sty}{\usepackage{xurl}}{} % add URL line breaks if available
\IfFileExists{bookmark.sty}{\usepackage{bookmark}}{\usepackage{hyperref}}
\hypersetup{
  pdftitle={Help},
  pdfauthor={Mohamed},
  hidelinks,
  pdfcreator={LaTeX via pandoc}}
\urlstyle{same} % disable monospaced font for URLs
\usepackage[margin=1in]{geometry}
\usepackage{graphicx,grffile}
\makeatletter
\def\maxwidth{\ifdim\Gin@nat@width>\linewidth\linewidth\else\Gin@nat@width\fi}
\def\maxheight{\ifdim\Gin@nat@height>\textheight\textheight\else\Gin@nat@height\fi}
\makeatother
% Scale images if necessary, so that they will not overflow the page
% margins by default, and it is still possible to overwrite the defaults
% using explicit options in \includegraphics[width, height, ...]{}
\setkeys{Gin}{width=\maxwidth,height=\maxheight,keepaspectratio}
% Set default figure placement to htbp
\makeatletter
\def\fps@figure{htbp}
\makeatother
\setlength{\emergencystretch}{3em} % prevent overfull lines
\providecommand{\tightlist}{%
  \setlength{\itemsep}{0pt}\setlength{\parskip}{0pt}}
\setcounter{secnumdepth}{-\maxdimen} % remove section numbering

\title{Help}
\author{Mohamed}
\date{23/12/2020}

\begin{document}
\maketitle

\hypertarget{introduction}{%
\subsection{Introduction}\label{introduction}}

This presentation is part of the Developing Data Products Coursera.org
course project submission.

It is an R Presentation generated with RStudio.

The Shiny application pitched by this presentation is at
\url{https://halici.shinyapps.io/Gapminder-Data-Visualization-using-Shiny-and-Plotly/}

The Shiny app source code is available at
\url{https://github.com/nihathalici/Developing-Data-Products-course--Assignment-Week-4-Shiny-Application-and-Reproducible-Pitch/tree/master/Shiny}

\hypertarget{application-overview}{%
\subsection{Application Overview}\label{application-overview}}

\begin{itemize}
\tightlist
\item
  The application is written in Shiny which is a web application
  framework for R
\item
  The source code consists of two files: server.R and ui.R
\item
  server.R includes the the server logic of a Shiny web application
\item
  ui.R includes the the user-interface definition, which uses the
  sidebarLayout template
\item
  The application is hosted on Rstudio's shiny server in the cloud
  (Shinyapps.io)
\end{itemize}

\hypertarget{how-it-works---i-the-application-contains-left-panel}{%
\subsection{How it works? - I \textbar{} The Application contains: Left
Panel}\label{how-it-works---i-the-application-contains-left-panel}}

\begin{itemize}
\tightlist
\item
  Label the main titel using a textInput
\item
  Change/determine the size of the plot points using a numericInput
\item
  Add a line of best fit using a checkboxInput
\item
  Change/determine the color of the plot points using a colourInput
\item
  Select data options using selectInput and sliderInput
\item
  download filtered data using a downloadButton
\end{itemize}

\hypertarget{how-it-works---ii-the-application-contains-main-panel}{%
\subsection{How it works? - II \textbar{} The Application contains: Main
Panel}\label{how-it-works---ii-the-application-contains-main-panel}}

There are two tabs in main panel as below:

\begin{itemize}
\tightlist
\item
  Plot: This displays plot for corresponding dataframe
\item
  Table: This displays a searchable-interactive Table for corresponding
  dataframe
\end{itemize}

\includegraphics{gapminder_using_shiny_1.png}

\hypertarget{ready-to-give-it-a-try}{%
\subsection{Ready to give it a try?}\label{ready-to-give-it-a-try}}

Use the Shiny app at
\url{https://halici.shinyapps.io/Gapminder-Data-Visualization-using-Shiny-and-Plotly/}

Get the app source code at
\url{https://github.com/nihathalici/Developing-Data-Products-course--Assignment-Week-4-Shiny-Application-and-Reproducible-Pitch}

\end{document}
